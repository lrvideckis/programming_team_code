% a4 paper gives us over 2000 more square mm per page, for a total of
% almost 5/6 of a page extra space
\newif\ifuseletter

%\useletterfalse
\uselettertrue

% if you need extra room to staple, adjust textheight and topmargin.

\ifuseletter
\documentclass[12pt,letterpaper,twocolumn,landscape]{article}
\textheight 8.25in
\textwidth 10.75in
\evensidemargin -0.875in
\oddsidemargin -0.875in
\topmargin -0.875in
\else
\documentclass[12pt,a4paper,twocolumn]{article}
\textwidth 200mm  % a4 paper is 210mm wide
\textheight 287mm % a4 paper is 297mm high
\evensidemargin -1in
\oddsidemargin -1in
\topmargin -1in
\addtolength{\evensidemargin}{5mm}
\addtolength{\oddsidemargin}{5mm}
\addtolength{\topmargin}{5mm}
\fi


\headheight 15pt
\headsep 4pt
\addtolength{\textheight}{-14pt}

\usepackage[T1]{fontenc}
\usepackage[utf8]{inputenc}
\usepackage{listings}
\usepackage{xcolor}
\usepackage{fancyhdr}

\pagestyle{fancy}
\fancyhf{}
\lhead{South Dakota School of Mines and Technology}
\rhead{Page \thepage}


\newlength{\lstlinewidth}
\setlength{\lstlinewidth}{0.5\textwidth}
\addtolength{\lstlinewidth}{-6pt}

% used to make sure that lstlisting aligns every character in a column
\newlength{\normalbasewidth}
\settowidth{\normalbasewidth}{\ttfamily m}
\newlength{\smallbasewidth}
\settowidth{\smallbasewidth}{\ttfamily\small m}
\newlength{\footnotebasewidth}
\settowidth{\footnotebasewidth}{\ttfamily\footnotesize m}
\newlength{\scriptbasewidth}
\settowidth{\scriptbasewidth}{\ttfamily\scriptsize m}

%% \makeatletter
%% \lstdefinestyle{mystyle}{
%%   basewidth=\footnotebasewidth,
%%   basicstyle=\ttfamily\lst@ifdisplaystyle\ttfamily\footnotesize\fi
%% }
%% \makeatother


\makeatletter
\lstdefinestyle{mystyle}{
  basewidth=\scriptbasewidth,
  basicstyle=\ttfamily\lst@ifdisplaystyle\ttfamily\scriptsize\fi
}
\makeatother

\makeatletter
\lst@CCPutMacro\lst@ProcessOther {"2D}{\lst@ttfamily{-{}}{-{}}}
\@empty\z@\@empty
\makeatother

% choose the colors you like below.

\lstset{
	language=C++,
  linewidth=\lstlinewidth,
  style=mystyle,           % set basic font and spacing
  captionpos=t,            % sets the caption position to top
  columns=fixed,
  mathescape=false,
  breaklines=true,         % sets automatic line breaking
  breakatwhitespace=true,  % automatic breaks should only happen at whitespace
  extendedchars=false,     % ASCII printing characters only
  % escapechar={`},          % Anything between backtics is escaped
  % escapeinside={{(*}{*)}},
  % frame=single,          % adds a frame around the code
  frame=tlrb,              % adds a frame around the code
  rulecolor=\color{black},
  xleftmargin=1pt,
  xrightmargin=1pt,
  % frameround=fttf,
  % framesep=3pt,
  % rulesep=3pt,
  framerule=0.5pt,
  % framexleftmargin=10pt,
  % framexrightmargin=10pt,
  % framextopmargin=10pt,
  % framexbottommargin=10pt,
  %%   %fillcolor=\color{SkyBlue},
  %%   %rulesepcolor=\color{SkyBlue},
  %keywordstyle=[1]\color{blue},      % keyword style
  %keywordstyle=[2]\color{cyan},      % keyword style
  %keywordstyle=[3]\color{orange},
  keywordstyle=\color{blue},      % keyword style
  stringstyle=\color{magenta},     % string literal style
  commentstyle=\color{gray}\itshape,
  morecomment=[l][\color{orange}]{\#},
  identifierstyle=\color{purple},
  %numberbychapter=true,
  numbers=none, % where to put the line-numbers; (none, left, right)
  %numbersep=15pt,                 % how far the line-numbers are from the code
  %numberstyle=\tiny\color{gray},  % the style that is used for the line-numbers
  postbreak={\mbox{$\hookrightarrow\space$}},
  showspaces=false, % show spaces everywhere adding particular underscores
  showstringspaces=false,         % underline spaces within strings only
  showtabs=false,  % show tabs within strings adding particular underscores
  stepnumber=1,    % the step between printed line-numbers.
  tabsize=4        % sets default tab character size to 8 spaces
}



\begin{document}

\lstlistoflistings
\newpage


\lstinputlisting[caption={\textbf{GEOMETRY}}]{null}

\lstinputlisting[language=C++, firstline=14, lastline=302, caption={Geometry}]{geometry/geometry.cpp}
\lstinputlisting[language=C++, firstline=10, lastline=207, caption={Geometry}]{geometry/geometry1.cpp}
\lstinputlisting[language=C++, firstline=5,  lastline=57,  caption={Convex Hull}]{geometry/convexHullInt.cpp}
\lstinputlisting[language=C++, firstline=5,  lastline=46,  caption={Pick's Theorem}]{geometry/picksTheorem.cpp}
\lstinputlisting[language=C++, firstline=7,  lastline=126, caption={Rectangle Union}]{geometry/rectangle_union.cpp}
\lstinputlisting[language=C++, firstline=0,  lastline=31,  caption={Closest Pair of Points}]{geometry/closestPairOfPoints.cpp}


\lstinputlisting[caption={\textbf{GRAPHS}}]{null}

\lstinputlisting[language=C++, firstline=0, lastline=70,  caption={Bipartite Matching}]{graphs/bipartite.cpp}
\lstinputlisting[language=C++, firstline=0, lastline=60,  caption={Bridges}]{graphs/bridges.cpp} \lstinputlisting[language=C++, firstline=0, lastline=60,  caption={Centroid Decomposition}]{graphs/centroid.cpp}
\lstinputlisting[language=C++, firstline=4, lastline=180, caption={Count Paths of Each Length}]{graphs/countPathLengths.cpp}
\lstinputlisting[language=C++, firstline=0, lastline=55,  caption={Cut Vertices}]{graphs/cutVertices.cpp}
\lstinputlisting[language=C++, firstline=0, lastline=22,  caption={Dijkstra's Algorithm}]{graphs/dijkstra.cpp}
\lstinputlisting[language=C++, firstline=0, lastline=59,  caption={DSU on Tree}]{graphs/dsuTree.cpp}
\lstinputlisting[language=C++, firstline=0, lastline=10,  caption={Floyd Warshall}]{graphs/floydWarshall.cpp}
\lstinputlisting[language=C++, firstline=0, lastline=47,  caption={Lowest Common Ancestor (LCA)}]{graphs/lca.cpp}
\lstinputlisting[language=C++, firstline=0, lastline=66,  caption={Strongly Connected Components (SCC)}]{graphs/scc.cpp}


\lstinputlisting[caption={\textbf{MAX FLOW}}]{null}
\lstinputlisting[language=C++, firstline=0, lastline=74,  caption={Dinic's Max Flow}]{maxflow/dinic.cpp}
\lstinputlisting[language=C++, firstline=0, lastline=48,  caption={Hungarian Algorithm}]{maxflow/hungarian.cpp}
\lstinputlisting[language=C++, firstline=0, lastline=75,  caption={Min Cost Max Flow}]{maxflow/minCostMaxFlow.cpp}


\lstinputlisting[caption={\textbf{STRINGS}}]{null}
\lstinputlisting[language=C++, firstline=0, lastline=64,  caption={KMP String Matching}]{stringAlgs/kmp.cpp}
\lstinputlisting[language=C++, firstline=0, lastline=27,  caption={Manachers (Palindromes)}]{stringAlgs/manachers.cpp}
\lstinputlisting[language=C++, firstline=0, lastline=37,  caption={Suffix Array}]{stringAlgs/suffixArray.cpp}
\lstinputlisting[language=C++, firstline=0, lastline=53,  caption={Trie}]{stringAlgs/trie.cpp}
\lstinputlisting[language=C++, firstline=0, lastline=23,  caption={Z Algorithm}]{stringAlgs/zAlg.cpp}


\lstinputlisting[caption={\textbf{DATA STRUCTURES}}]{null}
\lstinputlisting[language=C++, firstline=0, lastline=62,  caption={Longest Increasing Subsequence (LIS)}]{ds/LIS.cpp}
\lstinputlisting[language=C++, firstline=0, lastline=48,  caption={Count Rectangles}]{ds/cntRectangles.cpp}
\lstinputlisting[language=C++, firstline=0, lastline=88,  caption={Disjoint Set w/ Move}]{ds/disjointSet.cpp}
\lstinputlisting[language=C++, firstline=0, lastline=7,   caption={Indexed Set}]{ds/indexedSet.cpp}
\lstinputlisting[language=C++, firstline=0, lastline=30   caption={Largest Range Where Element Is Max/Min}]{ds/largestMinElementRange.cpp}
\lstinputlisting[language=C++, firstline=0, lastline=30   caption={Online Convex Hull}]{ds/onlineConvexHull.cpp}


\pagebreak
\end{document}
